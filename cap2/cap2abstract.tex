The environment in which agricultural activities are carried out presents high risk and great uncertainty. Several factors related to the agricultural sector can generate fluctuations in the income of producers. These fluctuations must be dealt with policies to support risk management, such as, for example, contracting rural insurance. This type of insurance makes it possible to recover the producer's financial capacity in the event of adverse events that cause economic loss. Considering the relevance of rural insurance in the agricultural sector, this work aims to evaluate the spatial distribution of the variables of this type of insurance in Brazilian municipalities in the period from 2006 to 2019. To achieve this objective, Principal Component Analysis was used with the objective of reduce the size of the data and the Exploratory Spatial Data Analysis was used to investigate the presence of spatial distribution patterns of rural insurance. The data used come from the Rural Insurance Census, compiled by the Ministry of Agriculture, Livestock and Supply. The results show that the highest concentrations of Rural Insurance policies are located in the South and Center-West regions and there is a tendency for an increase in the spatial dependence of Rural Insurance over the period analyzed. \\
\newline
\noindent {\textbf{Keywords}}: Rural insurance. Agricultural policy. Spatial statistics. Spatial autocorrelation. Moran's I.

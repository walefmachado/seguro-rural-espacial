O ambiente no qual se desenvolvem as atividades agropecuárias apresenta elevado risco e grande incerteza. Diversos fatores relacionados ao setor agropecuário podem gerar oscilações na renda dos produtores. Estas oscilações devem ser enfrentadas por meio de políticas de apoio à gestão de riscos como, por exemplo, a contratação de seguro rural. Esta modalidade de seguro possibilita a recuperação da capacidade financeira do produtor na ocorrência de eventos adversos que causem prejuízo econômico. Considerando a relevância do seguro rural no setor agropecuário, este trabalho tem como objetivo avaliar a distribuição espacial das variáveis desse seguro nos municípios brasileiros no período de 2006 a 2019. Para alcançar tal objetivo, foi utilizada a Análise de Componentes Principais (ACP) com o objetivo de reduzir a dimensionalidade dos dados e a Análise Exploratória de Dados Espaciais (AEDE) para investigar a presença de padrões de distribuição espacial do seguro rural. Os dados utilizados são provenientes dos censos do seguro rural, compilados pelo Ministério da Agricultura, Pecuária e Abastecimento (MAPA). Com a utilização de escores dos CPs, verificou-se que as maiores concentrações de apólices de seguro rural estão situadas nas regiões Sul e Centro-Oeste e há uma tendência de aumento na dependência espacial do seguro rural ao longo do período analisado. \\
\newline
\noindent {\textbf{Palavras-chave}}: Seguro rural. Política agrícola. Estatística espacial. Autocorrelação espacial. I de Moran.
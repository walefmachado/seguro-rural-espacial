%==========================================================================================

\usepackage[utf8]{inputenc}
\usepackage[brazil]{babel}
\usepackage[T1]{fontenc}
\usepackage{amsmath}
\usepackage{amsfonts}
\usepackage{mathrsfs}
\usepackage{amssymb}
\usepackage{graphicx}
\usepackage{geometry, calc, color, setspace}
\usepackage{indentfirst}
\usepackage{wrapfig}
\usepackage{boxedminipage}
\usepackage{enumerate}
\usepackage{float}
\usepackage[hang]{caption}
\usepackage{paralist}
\usepackage{comment}
\usepackage{icomma}
\usepackage{rotating}
\usepackage{multirow}
\usepackage[position=bottom]{subfig}
\usepackage{array}
\usepackage{tabularx}
\usepackage{float}
\usepackage{array}
\newcolumntype{L}[1]{>{\raggedright\let\newline\\\arraybackslash\hspace{0pt}}m{#1}}
\newcolumntype{C}[1]{>{\centering\let\newline\\\arraybackslash\hspace{0pt}}m{#1}}
%\newcolumntype{R}[1]{>{\raggedleft\let\newline\\\arraybackslash\hspace{0pt}}m{#1}}
\newcolumntype{R}{>{\raggedleft\let\newline\\\arraybackslash\hspace{0pt}}X}

%\usepackage[abnt-substyle=UFLA,abnt-emphasize=bf,abnt-etal-list=3,abnt-and-type=e]{abntcite}
\usepackage[alf,bibjustif]{abntex2cite}
%\usepackage{abntcite}

% Para o alinhamento dos títulos das figuras
%\captionsetup[subfigure]{labelfont=bf,textfont=normalfont,singlelinecheck=off,justification=raggedright}

%%  Inserindo os códigos Python ==================================
\usepackage{listings}

%\definecolor{light_gray}{rgb}{0.97,0.97,0.97}
%\definecolor{mymauve}{rgb}{0.58,0,0.82}
%\definecolor{mygreen}{rgb}{0,0.6,0}

\lstset{
  language = Python,
  inputencoding = utf8,
  backgroundcolor = \color{white},
  columns=fullflexible,
  basicstyle=\ttfamily,
  breaklines=true,
  postbreak=\raisebox{0ex}[0ex][0ex]{\color{black}$\hookrightarrow$\space},
  keywordstyle=\color{black},      % keyword style
  stringstyle=\color{black},
  commentstyle=\color{black}
}

\usepackage[labelsep=endash]{caption}

\newcommand{\HRule}{\noindent\rule{\linewidth}{0.2mm}}

\usepackage{mathpazo}                         % tem suporte matemático
\usepackage[scaled=0.85]{beramono}            % usa esta nos verbatins [scaled=0.9]

\renewcommand\UrlFont{\color{black}\rmfamily} 

\def\distnumber{2.3em}

%==========================================================================================

\author{Walef Machado de Mendonça\footnote{Mestrando em Estatística Aplicada e Biometria na Universidade Federal de Alfenas}}

%==========================================================================================

%==========================================================================================
% Pacotes

%% definições relacionadas a idioma
\usepackage[brazil]{babel}

%% codificação de caracteres
%% http://tex.stackexchange.com/questions/664/why-should-i-use-usepackaget1fontenc
%% http://tex.stackexchange.com/questions/44694/fontenc-vs-inputenc
\usepackage[T1]{fontenc}
\usepackage[utf8]{inputenc}    % usar se salvar com codificação UTF-8
%\usepackage[latin1]{inputenc}  % usar se salvar com codificação ISO

%% definições das margens
\usepackage[top=3cm,left=3cm,right=2cm,bottom=2cm]{geometry}

%% para inserir figuras em diversas extensões
\usepackage{graphicx}

%% controla o uso de Teorema, Definição, Exemplo, Demonstração, Exercício
\usepackage{amsthm}

%% todos os pacotes da AMS (American Mathematical Society), inúmeros símbolos
\usepackage{amsfonts,amssymb,amsxtra,empheq}  
\usepackage[mathscr]{eucal}

%% permite o uso de vírgula como separador decimal no ambiente matemático
\usepackage{icomma} 

%% permite o uso de comentários entre \begin{comment}...\end{comment}                 
\usepackage{comment}

%% cores em palavras, frases, etc. \textcolor{red}{text in red}
\usepackage{color}

%% permite a indentação do 1º paragráfo após um título de seção, capítulo
\usepackage{indentfirst}
%%%% recuo da primeira linha de cada parágrafo
  %\setlength{\parindent}{1.2cm}

%% permite a construção do texto colunas
\usepackage{multicol}

%% controla o espaçamento entre linhas
%\usepackage{setspace}
%%%% espaçamento 1.5 linha no texto \begin{spacing}{1.0}...\end{spacing}
%\onehalfspace   

%% permite desenhar
%% http://www.texample.net/tikz/
%\usepackage{tikz}
%\usetikzlibrary{decorations.pathreplacing}
%\usetikzlibrary{arrows,decorations.pathmorphing}
%\usetikzlibrary{shapes,backgrounds}  
  
%% para colocar a legenda indentada como formato da UFLA
\usepackage[format=hang, labelsep=quad]{caption}

%% controla objetos flutuantes e permite definir novas classes
%% útil para criar a classe de fluantes 'program', para código
\usepackage{float}
%%%% algumas formas de denifir
%   \floatstyle{ruled}
%   \newfloat{Input}{tbp}{lop}[section]
%   \floatname{Input}{Program}

%% controla novos ambientes que listam índices, e.g. lista de anexos
\usepackage{listings}

%% texto em caixas
\usepackage{boxedminipage}

%% controle e definição de ambientes numerados
%% listas mais compactas, úteis para salvar espaço
%% http://tex.stackexchange.com/questions/62009/paralist-environment
\usepackage{paralist}

%% para rotacionar tabela \begin{sidewaystable}
\usepackage{rotating} 

%% rotacionar texto \begin{landscape}
\usepackage{lscape} 

%% para ter tabelas com linhas mescladas
\usepackage{multirow}

%% para inserir epigrafe
\usepackage{epigraph}

%% tabelas 
\usepackage{tabularx}

%% notas de rodapé
\usepackage[symbol, hang, flushmargin]{footmisc} 

% controle e definições para tamanho de colunas em tabelas
\usepackage{array} 
%\newcolumntype{C}[1]{>{\centering\let\newline\\\arraybackslash\hspace{0pt}}m{#1}}
%\newcolumntype{R}[1]{>{\raggedleft\let\newline\\\arraybackslash\hspace{0pt}}m{#1}}
%\newcolumntype{R}{>{\raggedleft\let\newline\\\arraybackslash\hspace{0pt}}X}

\newcolumntype{L}[1]{>{\raggedright\let\newline\\\arraybackslash\hspace{0pt}}m{#1}}
\newcolumntype{C}[1]{>{\centering\let\newline\\\arraybackslash\hspace{0pt}}m{#1}}
%\newcolumntype{R}[1]{>{\raggedleft\let\newline\\\arraybackslash\hspace{0pt}}m{#1}}
\newcolumntype{R}{>{\raggedleft\let\newline\\\arraybackslash\hspace{0pt}}X}


\usepackage{tocbasic}

% para a lista de figuras 
\DeclareTOCStyleEntry[
  entrynumberformat=\entrynumberwithprefix{\figurename},
  dynnumwidth,
  numsep=1em
]{tocline}{figure}
\newcommand\entrynumberwithprefix[2]{#1\enspace#2  -\hfill}

% para lista de tabelas
\DeclareTOCStyleEntry[
  entrynumberformat=\entrynumberwithprefix{\tablename},
  dynnumwidth,
  numsep=1em
]{tocline}{table}
\newcommand\entrynumberwithprefixd[2]{#1\enspace#2  -\hfill}

%==========================================================================================
% Definições

%% redefine o espaçamento entre linhas
\renewcommand{\baselinestretch}{1.2cm}
\def\onehalspacing{1.5}


%% permite leitura do caracter @ nas definições que seguem
\makeatletter

%%.....................................................
%% Sessões

%% definição do texto título da sessão
\renewcommand{\section}{\@startsection
{section}%                     % the name
{1}%                           % the level
{0mm}%                         % the indent
{-\baselineskip}%              % the before skip
{1.\baselineskip}%             % the after skip
{\noindent\normalsize\textbf}} % the style

%% definição do texto título da subsessão
\renewcommand{\subsection}{\@startsection
{subsection}%                  % the name
{2}%                           % the level
{0mm}%                         % the indent
{-\baselineskip}%              % the before skip
{0.1\baselineskip}%            % the after skip
{\noindent\normalsize}} % the style

%% definição do texto título da subsubsessão
\renewcommand{\subsubsection}{\@startsection
{subsubsection}%                  % the name
{2}%                           % the level
{0mm}%                         % the indent
{-\baselineskip}%              % the before skip
{0.1\baselineskip}%            % the after skip
{\noindent\normalsize}} % the style


%\makeatother


%% desabilita o uso do @
\makeatother

%%.....................................................
%% Sumário

%% sumário alinhado à esquerda
\usepackage{titletoc}
\def\distnumber{2.3em}

\titlecontents{section}       % set formatting for \section
[\distnumber]                 % adjust left margin
{\bfseries}                   % font formatting
{\contentslabel{\distnumber}} % section label and offset
{\hspace*{-\distnumber}}
{\normalfont\titlerule*[1pc]{.}\contentspage}

\titlecontents{subsection}    % set formatting for \subsection
[\distnumber]                 % adjust left margin
{\normalfont}                   % font formatting
{\contentslabel{\distnumber}} % section label and offset
{\hspace*{-\distnumber}}
{\normalfont\titlerule*[1pc]{.}\contentspage}

\titlecontents{subsubsection} % set formatting for \subsubsection
[\distnumber]                 % adjust left margin
{\bfseries}                   % font formatting
{\contentslabel{\distnumber}} % section label and offset
{\hspace*{-\distnumber}}
{\normalfont\titlerule*[1pc]{.}\contentspage}

\setcounter{tocdepth}{4}
\renewcommand{\baselinestretch}{1.5} % espaçamento entre linha
\numberwithin{equation}{section}     % numerar equação por seção

%%.....................................................
%% Flutuantes e equações

%% nomes
\renewcommand{\figurename}{Figura}
\renewcommand{\tablename}{Tabela}

%% profundidade de numeração
%\numberwithin{figure}{section}
%\numberwithin{table}{section}
%\numberwithin{equation}{section}

%%.....................................................
%% Paginação

\usepackage{fancyhdr}
\renewcommand{\headrulewidth}{0pt}
\fancyhf{}
\fancyhead[R]{\thepage}

%%.....................................................
%% Fontes

%% fonte para o texto
\usepackage{times}

%% fonte para ambiente matemático mais parecido com Times
\usepackage{txfonts}

%% fonte para ambiente de código (monospaced)
\usepackage[scaled=0.85]{beramono}

%%.....................................................
%% Referências e hyperlinks

\usepackage{hyperref} %transformas o sumário e as citações em hiperlinks
\hypersetup{
%    bookmarks=true,         % show bookmarks bar?
%    unicode=false,          % non-Latin characters in Acrobat's bookmarks
%    pdftoolbar=true,        % show Acrobat's toolbar?
%    pdfmenubar=true,        % show Acrobat's menu?
%    pdffitwindow=true,      % page fit to window when opened
  pdftitle={O seguro rural no Brasil},
  pdfauthor={Walef Machado de Mendonça},
  pdfsubject={Dissertação de Mestrado},
  pdfcreator={Walef Machado de Mendonça},
  pdfproducer={Walef Machado de Mendonça},
  %pdfkeywords={Verossimilhança, Curvatura, Método delta, van Genuchten},
%    pdfnewwindow=true,      % links in new window
  colorlinks=true,           % false: boxed links; true: colored links
  linkcolor=black,           % color of internal links
  citecolor=black,           % color of links to bibliography
  filecolor=red,             % color of file links
  urlcolor=cyan              % color of external links
}

% Para multiplas figuras
\usepackage[position=bottom]{subfig}
\usepackage[labelsep=endash]{caption}

\usepackage[alf,bibjustif]{abntex2cite}

%%.....................................................
%% Comandos

\newcommand{\lin}{\noindent \rule[4mm]{\textwidth}{0.1ex}}

\renewcommand\UrlFont{\color{black}\rmfamily} 


%%  Inserindo os códigos Python ==================================
\usepackage{listings}

\definecolor{light_gray}{rgb}{0.97,0.97,0.97}
\definecolor{mymauve}{rgb}{0.58,0,0.82}
\definecolor{mygreen}{rgb}{0,0.6,0}
\lstset{
  basicstyle=\ttfamily,
  inputencoding = utf8,
  language = Python,
  backgroundcolor = \color{white},
  columns=fullflexible,
  breaklines=true,
  postbreak=\raisebox{0ex}[0ex][0ex]{\color{red}$\hookrightarrow$\space},
  keywordstyle=\color{black},      % keyword style
  stringstyle=\color{black},
  commentstyle=\color{black}
}


%==========================================================================================

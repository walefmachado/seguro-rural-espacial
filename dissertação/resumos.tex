%=========================================================================================================
% Resumo

\begin{singlespace}
\begin{center}
\section*{RESUMO}
\end{center}
As atividades agropecuárias se inserem em um contexto de adversidades que as colocam em situação diferenciada em relação aos riscos enfrentados pelos produtores. Tais atividades demandam grandes investimentos, o que faz com que sua atratividade esteja relacionada às formas existentes de gerenciamento de riscos. Uma das formas usuais de gerenciamento de risco neste setor é a contratação de seguro rural, uma vez que esta modalidade de seguro possibilita a recuperação da capacidade financeira do produtor na ocorrência de sinistros. Dessa forma, o objetivo da presente pesquisa é analisar a distribuição e a dinâmica espacial de variáveis relacionadas às apólices de seguro rural contratadas nos municípios brasileiros no período de 2006 a 2019. Para tanto, a dissertação foi dividida em dois artigos. No primeiro, o objetivo é avaliar a evolução das variáveis relacionadas ao seguro rural de forma individual. Para esse fim, foi realizada uma análise descritiva da evolução das variáveis de apólices de seguro rural. Os resultados indicam que as maiores concentrações de apólices de seguro rural estão situadas nas regiões Sul, Sudeste e Centro-Oeste. No segundo artigo, o objetivo é investigar a distribuição espacial do seguro rural nos municípios brasileiros no período de 2006 a 2019. Para alcançar tal objetivo, foi utilizada a Análise de Componentes Principais, com objetivo de reduzir a dimensionalidade dos dados. Os escores do primeiro componente principal foram  utilizados na Análise Exploratória de dados Espaciais para investigar a presença de padrões de distribuição espacial de dados multivariados de seguro rural. Os resultados apontam que as maiores concentrações de apólices de seguro rural estão situadas nas regiões Sul, Sudeste e Centro-Oeste. Além disso, apesar de haver um aumento nas contratações de seguro rural, há também uma tendência de maior concentração espacial das apólices ao longo do período analisado. Em ambos os artigos foram utilizados dados dos Censos do seguro rural, compilados pelo Ministério da Agricultura, Pecuária e Abastecimento e dados com atributos geográficos do território disponíveis no endereço eletrônico do Instituto Brasileiro de Geografia e Estatística.\\
\newline
\noindent Palavras-chave: Seguro rural; Política agrícola; Estatística espacial; Autocorrelação espacial; \hspace*{6.4em} I de Moran.
\end{singlespace}


%=========================================================================================================
% Abstract

\newpage
\begin{singlespace}
\begin{center}
\section*{ABSTRACT}
\end{center}
Agricultural activities are part of a context of adversities that place them  in a different situation in relation to the risks faced by the producers. Such activities require large investments, which makes their attractiveness related to existing forms of risk management. One of the usual forms of risk management in this sector is the contracting of rural insurance, since this type of insurance makes it possible to recover the financial capacity of the producer in the event of accidents. In this way, the objective of this research is to analyze the distribution and spatial dynamics  of variables related to rural insurance policies contracted in Brazilian municipalities from 2006 to 2019. To this end, the dissertation was divided into two articles. In the first, the objective is to evaluate the evolution of variables related to rural insurance individually. To this end, a descriptive analysis of the evolution of rural insurance policy variables was carried out. The results indicate that the highest concentrations of rural insurance policies are located in the South, Southeast and Midwest regions. In the second article, the objective is to investigate the spatial distribution of rural insurance in Brazilian municipalities in the period from 2006 to 2019. To achieve this objective, Principal Component Analysis was used, in order to reduce the dimensionality of the data. The scores of the first principal component were used in Exploratory Spatial Data Analysis to investigate the presence of spatial distribution patterns in multivariate rural insurance data. The results indicate that the highest concentration of rural insurance policies are located in the South, Southeast and Midwest regions. In addition, despite an increase in rural insurance contracts, there is also a trend of greater spatial concentration of policies over  the period. In both articles data from the Rural Insurance Census were used, compiled by the Ministry of Agriculture, Livestock and Supply, and data with geographic attributes of the territory available at the website of the Brazilian Institute of Geography and Statistics.\\
\newline
\noindent {Keywords:} Rural insurance; Agricultural policy; Spatial statistics; Spatial autocorrelation; Mo-\\
\hspace*{4.5em} ran’s I.

\end{singlespace}

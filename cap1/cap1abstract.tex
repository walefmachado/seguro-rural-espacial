Agricultural activities are inserted in a context of adversities that place them in a different situation in relation to the risks faced by producers. Such activities demand large investments, which makes their attractiveness related to existing forms of risk management. One of the most common forms of risk management in this sector is the contracting of Rural Insurance, since this type of insurance makes it possible to recover the producer's financial capacity in the event of accidents. In this sense, the objective of this work is to evaluate the distribution and spatial dynamics of variables related to Rural Insurance policies contracted in Brazilian municipalities from 2006 to 2019. In addition, it seeks to investigate the existence of spatial dependence and the presence of groups with a large number of Rural Insurance policies. For this purpose, data from the Rural Insurance Census, compiled by the Ministry of Agriculture, Livestock and Supply (MAPA) and Exploratory Analysis of Spatial Data (AEDE) are used. The results show that the highest concentrations of Rural Insurance policies are located in the South and Center-West regions. In addition, despite an increase in Rural Insurance contracts, there is also a trend towards greater spatial concentration of policies over the analyzed period . \\
\newline
\noindent {\textbf{Keywords}}: Rural insurance. Agricultural policy. Spatial statistics. Spatial autocorrelation. Moran's I. 

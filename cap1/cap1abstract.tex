\noindent Agricultural activities demand large investments. Therefore, the context of adversities, in which these activities are inserted, presents a risk scenario that encourages producers to search for existing forms of risk management. One of the most common forms of risk management in this sector is the contracting of Rural Insurance, since this type of insurance makes it possible to recover the financial capacity of the producer in the event of claims. In this sense, the objective of this work is to evaluate the evolution and the spatial dynamics of variables related to Rural Insurance policies contracted in Brazilian municipalities from 2006 to 2019. For this, data were taken from the Rural Insurance Census, compiled by the Ministry of Agriculture, Livestock and Supply. The results indicate that the highest concentrations of Rural Insurance policies are located in the South, Southeast and Midwest regions. In addition, despite an increase in Rural Insurance contracts, there is also a trend of greater spatial concentration of policies over the analyzed period. \\ 
\newline
\noindent {\textbf{Keywords}}:  Rural insurance. Agricultural policy. Agricultural risk management. Subvention.

As atividades agropecuárias se inserem em um contexto de adversidades que as colocam em situação diferenciada em relação aos riscos enfrentados pelos produtores. Tais atividades demandam grandes investimentos, o que faz com que sua atratividade esteja relacionada às formas existentes de gerenciamento de riscos. Uma das formas mais usuais de gerenciamento de risco neste setor é a contratação de Seguro Rural, uma vez que esta modalidade de seguro possibilita a recuperação da capacidade financeira do produtor na ocorrência de sinistros. Nesse sentido, o objetivo do trabalho é avaliar a distribuição e a dinâmica espacial de variáveis relacionadas às apólices de Seguro Rural contratadas nos municípios brasileiros no período de 2006 a 2019. Além disso, busca-se investigar a existência de dependência espacial e a presença de agrupamentos com grande número de apólices de Seguro Rural. Para tanto, utiliza-se os dados dos Censos do Seguro Rural, compilados pelo Ministério da Agricultura, Pecuária e Abastecimento (MAPA) e Análise Exploratória de Dados Espaciais (AEDE). Os resultados apontam que as maiores concentrações de apólices de Seguro Rural estão situadas nas regiões Sul e Centro-Oeste. Além disso, apesar de haver um aumento nas contratações de Seguro Rural, há também uma tendência de maior concentração espacial das apólices ao longo do período analisado. \\
\newline
\noindent {\textbf{Palavras-chave}}: Seguro rural. Política agrícola. Estatística espacial. Autocorrelação espacial. I de Moran.
\noindent As atividades agropecuárias demandam grandes investimentos. Portanto, o contexto de adversidades, no qual estas atividades estão inseridas, apresenta um cenário de riscos que estimula nos produtores a busca pelas formas existentes de gerenciamento de risco. Uma das formas mais usuais de gerenciamento de risco neste setor é a contratação de Seguro Rural, uma vez que esta modalidade de seguro possibilita a recuperação da capacidade financeira do produtor na ocorrência de sinistros. Nesse sentido, o objetivo do trabalho é avaliar a evolução e a dinâmica espacial de variáveis relacionadas às apólices de Seguro Rural contratadas nos municípios brasileiros no período de 2006 a 2019. Para tanto, utilizam-se os dados dos Censos do Seguro Rural, compilados pelo Ministério da Agricultura, Pecuária e Abastecimento. Os resultados apontam que as maiores concentrações de apólices de Seguro Rural estão situadas nas regiões Sul, Sudeste e Centro-Oeste. Além disso, apesar de haver um aumento nas contratações de Seguro Rural, há também uma tendência de maior concentração espacial das apólices ao longo do período analisado. \\
\newline
\noindent {\textbf{Palavras-chave}}: Política agrícola. Gestão de riscos agrícolas. Subvenção. Regiões. 